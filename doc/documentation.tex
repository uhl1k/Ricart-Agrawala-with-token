\documentclass[11pt]{scrartcl}

\usepackage{hyperref}

\title{Ricart-Agrawala algoritmus v Javě}
\subtitle{Semestrální práce do DSV na FEL ČVUT v Praze}
\author{Roman Janků \\ (jankurom@fel.cvut.cz)}
\date{29. prosince 2021}

\begin{document}

    \maketitle

    Tento dokument popisuje funkcionalitu projektu, který jsem vytvořil pro předmět B2M32DSVA Distribuované systémy a
    výpočty na Fakultě Elektrotechnické Českého Vysokého Učení Technického v Praze v zimním semestru 2021/2022.

    Projekt se zabývá implementací Ricart-Agrawala algoritmu v Javě za využití Java RMI\footnote{Java Remote Method
    Invocation}. Zdrojové kódy projektu jsou dostupné na fakultním GitLabu
    \footnote{\url{https://gitlab.fel.cvut.cz/jankurom/dsv-semestral-work}}.

    Vše důležité se odehrává ve třídě \texttt{Server}. Tato třída si pamatuje všechny nody v síti a v ní je
    naimplementovaný Ricart-Agrawala algoritmus. Ovládat node lze pomocí veřejných metod na této třídě:

    \begin{itemize}
        \item \texttt{getInstance()} vrací jedinou instanci třídy \texttt{Server}, protože tato třída je singleton.
        \item \texttt{start()} nastartuje node a připojí ho do sítě. Pokud je node už nastartovaný, je vyvolána
        výjimka. Při spouštění se node zeptá na adresu a pak na této adrese vytvoří RMI registr. Pak se zeptá, jestli
        se má připojit do již existující sítě. Pokud ano, zeptá se na adresu libovolného nodu v této síti a připojí
        se k němu. Po zavolání metody \texttt{connecting()} získá seznam všech nodů v síti a připojí se k nim.
        \item \texttt{getUuid()} returns UUID of this node.
        \item \texttt{drop()} terminates node without notifying other nodes in network. It also resets node to 
        initial state.
        \item \texttt{stop()} informuje všechny nody v síti, že končí a pak zavolá metodu \texttt{drop()}.
        \item \texttt{editVariable()} spustí editaci sdílené proměnné. Pokud node nemá token, rozešle žádost všem 
        ostatním nodům, počká na odpověď, vstoupí do kritické sekce, upraví proměnnou a předá token dál (pokud je komu).
        \item \texttt{receivedRequest()} je voláno, když node přijme žádost od jiného nodu. Pokud má node token, tak 
        ho předá. Jinak si jen zapamatuje žádost.
        \item \texttt{isRunning()} vrátí TRUE, když node běží a FALSE když ne.
        \item \texttt{getRemotes()} vrátí seznam všech nodů v síti.
        \item \texttt{addRemote()} přidá node do seznamu a připojí se k němu.
        \item \texttt{hasToken()} vrátí TRUE, když node má token, jinak vrátí FALSE.
        \item \texttt{generateToken()} vygeneruje token na tomto nodu.
        \item \texttt{removeRemote()} odstraní node ze seznamu a odpojí se od něho.
    \end{itemize}

    Pokud je během provádění libovolné metody kroku detekován node, který po zavolání metody přes RMI vrátí Remote
    exception, je tento node vyřazen ze seznamu a všichni jsou o této chybě informováni. Tento node se totiž s
    největší pravděpodobností odpojil ze sítě, aniž by dal ostatním nodům v síti vědět.

    Třída \texttt{Remote} je jakýmsi obalem, který obsahuje adresu nodu, odkaz na jeho implementaci, UUID nodu, a
    hodiny, kdy node žádal o token a kdy ho dostal. Pro hezký výpis je implementována metoda \texttt{toString} a pro
    účely řazení je implementovány metody \texttt{compareTo()} a \texttt{equals()}.

    \texttt{DsvStub} a \texttt{DsvImplementation} slouží pro komunikaci pomocí Java RMI. Naprogramoval jsem
    následující metody:

    \begin{itemize}
        \item \texttt{disconnecting()} volá node, když se dopojuje ze sítě, aby informoval všechny ostatní v síti.
        \item \texttt{connecting()} volá node při připojování do sítě. Informuje tím o svém připojení a získává
        seznam všech nodů v síti.
        \item \texttt{getUUID()} získá UUID volaného nodu.
        \item \texttt{request()} je používán Ricart-Agrawala algoritmem pro žádání o token. Tato metoda volá
        \texttt{receivedRequest()} na serveru, který předá token (pokud nějaký má).
        \item \texttt{token()} slouží k předání token v algoritmu.
    \end{itemize}

    Třída \texttt{Main} obsahuje metodu \texttt{main()}, která spouští node a také obsahuje interface k serveru.
    Třídy \texttt{AlreadyStartedException} a \texttt{AlreadyStoppedException} jsou vyvolávány serverem, pokud dojde k
    chybě.

\end{document}